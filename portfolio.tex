\documentclass[12pt]{report}

% Packages
\usepackage{geometry}        % Adjust margins
\usepackage{graphicx}        % Include graphics
\usepackage{amsmath}         % Advanced math typesetting
\usepackage{booktabs}        % Professional-quality tables
\usepackage{hyperref}        % Hyperlinks
\usepackage{fancyhdr}        % Custom headers and footers
\usepackage{longtable}       % Tables that span multiple pages
\usepackage{titlesec}        % Customize section titles
\usepackage{tocloft}         % Customize table of contents
\usepackage{chngcntr}        % Change counter for figures and tables

% Page Setup
\geometry{
    a4paper,
    left=25mm,
    right=25mm,
    top=25mm,
    bottom=25mm,
}

% Header and Footer
\pagestyle{fancy}
\fancyhf{}
\lhead{Long-Term Value Investing Portfolio}
\rhead{\thepage}
\renewcommand{\headrulewidth}{0.4pt}

% Section Formatting
\titleformat{\chapter}{\LARGE\bfseries}{\thechapter}{1em}{}
\titleformat{\section}{\large\bfseries}{\thesection}{1em}{}
\titleformat{\subsection}{\normalsize\bfseries}{\thesubsection}{1em}{}

% Table of Contents Formatting
\renewcommand{\cftchapleader}{\cftdotfill{\cftdotsep}}
\setlength{\cftbeforechapskip}{1.0em}

% Reset figure and table counters for each chapter
\counterwithin{figure}{chapter}
\counterwithin{table}{chapter}

% Document
\begin{document}

% Title Page
\begin{titlepage}
    \centering
    \vspace*{2cm}
    {\huge\bfseries Long-Term Value Investing Portfolio Analysis\par}
    \vspace{1.5cm}
    {\Large Your Name\par}
    \vfill
    {\large \today\par}
\end{titlepage}

% Abstract
\begin{abstract}
\noindent
This report presents a comprehensive analysis of a concentrated, long-term value investing portfolio comprising Amazon.com, Inc. (AMZN), Alibaba Group Holding Limited (BABA), and ASML Holding N.V. (ASML). The analysis includes an in-depth examination of each company's business model, financial performance, intrinsic valuation, and associated risks, supporting the investment thesis for each.
\end{abstract}

% Table of Contents
\tableofcontents
\newpage

% Introduction
\chapter{Introduction}
\section{Investment Philosophy}
Our investment philosophy centers on long-term value investing, focusing on companies with strong fundamentals, competitive advantages, and potential for sustainable growth. We aim to identify undervalued stocks that offer significant upside potential over an extended investment horizon.

\section{Portfolio Overview}
This concentrated portfolio includes three leading companies in their respective industries: Amazon.com, Inc., Alibaba Group Holding Limited, and ASML Holding N.V. These companies have been selected based on their robust business models, financial strength, and growth prospects.

% Stock Analyses
\chapter{Calculation Metrics}

% Amazon Analysis
\chapter{Amazon.com, Inc. (AMZN)}
\section{Company Overview}
Amazon.com, Inc. is a global leader in e-commerce and cloud computing services. Founded in 1994 by Jeff Bezos, Amazon has expanded from an online bookstore to a diversified technology company offering a wide range of products and services, including Amazon Web Services (AWS), digital streaming, and artificial intelligence.

\section{Industry Analysis}
Amazon operates primarily in the e-commerce and cloud computing industries. The e-commerce sector continues to grow, driven by increasing internet penetration and consumer preference for online shopping. AWS dominates the cloud computing market, which is expected to expand due to the rising demand for cloud-based solutions.

\section{Investment Thesis}
Amazon's diversified revenue streams, market leadership, and continuous innovation position it for sustained growth. The company's strong cash flows from its e-commerce segment fund investments in high-margin businesses like AWS and advertising, enhancing shareholder value.

\section{Financial Analysis}
\subsection{Historical Financial Performance}
From 2018 to 2022, Amazon's revenue grew at a compound annual growth rate (CAGR) of approximately 24\%. Net income experienced volatility due to significant reinvestment and market conditions but generally trended upward until 2021.

\subsection{Key Financial Ratios}
\begin{itemize}
    \item \textbf{Price-to-Earnings (P/E) Ratio}: As of October 2023, Amazon's P/E ratio is high due to its reinvestment strategy.
    \item \textbf{Return on Equity (ROE)}: The ROE has been modest, reflecting significant equity growth from retained earnings.
    \item \textbf{Debt-to-Equity Ratio}: Amazon maintains a reasonable debt-to-equity ratio, indicating prudent financial leverage.
\end{itemize}

\section{Intrinsic Valuation}
\subsection{Discounted Cash Flow (DCF) Analysis}
A DCF analysis estimates Amazon's intrinsic value by projecting free cash flows over the next ten years, assuming a revenue growth rate of 15\% gradually declining to 3\% in perpetuity, and discounting at a weighted average cost of capital (WACC) of 8\%. The analysis suggests that Amazon is undervalued relative to its current market price.

\subsection{Comparable Company Analysis}
Comparing Amazon to peers like Alphabet Inc. and Microsoft Corporation indicates that Amazon's valuation multiples are justified given its growth prospects and market position.

\section{Risk Analysis}
Key risks include increased competition in e-commerce and cloud services, regulatory challenges, and potential economic downturns affecting consumer spending.

\section{Conclusion}
Amazon's strong market position, diversified business model, and growth potential make it an attractive long-term investment for a value-focused portfolio.

% Alibaba Analysis
\chapter{Alibaba Group Holding Limited (BABA)}
\section{Company Overview}
Alibaba Group Holding Limited is China's largest e-commerce company, founded in 1999 by Jack Ma. It operates various online marketplaces, including Taobao and Tmall, and offers services in cloud computing, digital media, and fintech.

\section{Industry Analysis}
Alibaba operates in China's rapidly growing e-commerce and cloud computing sectors. Despite regulatory challenges, the long-term outlook remains positive due to increasing consumer spending and digitalization in China.

\section{Investment Thesis}
Alibaba's dominant market share in Chinese e-commerce and its expanding cloud computing services position it well for future growth. The company's investments in technology and international expansion offer additional growth avenues.

\section{Financial Analysis}
\subsection{Historical Financial Performance}
Between 2018 and 2022, Alibaba's revenue grew at a CAGR of approximately 35\%. However, net income has been impacted by regulatory fines and increased investments.

\subsection{Key Financial Ratios}
\begin{itemize}
    \item \textbf{P/E Ratio}: Alibaba's P/E ratio is lower than industry peers, suggesting potential undervaluation.
    \item \textbf{ROE}: The company has maintained a strong ROE, reflecting efficient use of equity capital.
    \item \textbf{Debt-to-Equity Ratio}: Alibaba has a low debt-to-equity ratio, indicating a strong balance sheet.
\end{itemize}

\section{Intrinsic Valuation}
\subsection{Discounted Cash Flow (DCF) Analysis}
Assuming a revenue growth rate of 20\% declining to 3\% over ten years and a WACC of 9\%, the DCF model indicates that Alibaba is significantly undervalued.

\subsection{Comparable Company Analysis}
Compared to peers like JD.com and Tencent Holdings, Alibaba trades at lower multiples, presenting a potential investment opportunity.

\section{Risk Analysis}
Risks include regulatory pressures from the Chinese government, geopolitical tensions affecting international operations, and increased competition.

\section{Conclusion}
Despite regulatory challenges, Alibaba's strong market position and growth prospects make it a compelling investment for long-term value investors.

% ASML Analysis
\chapter{ASML Holding N.V. (ASML)}
\section{Company Overview}
ASML Holding N.V. is a Dutch company that designs and manufactures photolithography systems used in the semiconductor industry. It is the world's leading supplier of extreme ultraviolet (EUV) lithography equipment.

\section{Industry Analysis}
ASML operates in the semiconductor equipment industry, which is critical for the production of advanced microchips. The industry is poised for growth due to increasing demand for semiconductors in various applications like AI, 5G, and IoT.

\section{Investment Thesis}
ASML's technological leadership and monopolistic position in EUV lithography provide significant competitive advantages. The company's strong order backlog and the essential nature of its equipment for chipmakers suggest robust future growth.

\section{Financial Analysis}
\subsection{Historical Financial Performance}
From 2018 to 2022, ASML's revenue grew at a CAGR of around 18\%, with consistent improvements in net income and operating margins.

\subsection{Key Financial Ratios}
\begin{itemize}
    \item \textbf{P/E Ratio}: ASML's P/E ratio is higher than industry averages, reflecting its growth prospects.
    \item \textbf{ROE}: The company boasts a high ROE, indicating efficient use of shareholder capital.
    \item \textbf{Debt-to-Equity Ratio}: ASML maintains a low debt-to-equity ratio, suggesting financial stability.
\end{itemize}

\section{Intrinsic Valuation}
\subsection{Discounted Cash Flow (DCF) Analysis}
Using a projected revenue growth rate of 15\% declining to 3\% over ten years and a WACC of 7.5\%, the DCF analysis shows that ASML is fairly valued to slightly undervalued.

\subsection{Comparable Company Analysis}
When compared to competitors like Applied Materials and Lam Research, ASML's premium valuation is justified by its unique market position.

\section{Risk Analysis}
Risks include cyclical demand in the semiconductor industry, geopolitical tensions affecting supply chains, and technological disruptions.

\section{Conclusion}
ASML's dominant position and essential role in semiconductor manufacturing make it a valuable addition to a long-term value portfolio.

% Portfolio Analysis
\chapter{Portfolio Analysis}
\section{Aggregate Portfolio Metrics}
The portfolio exhibits a balanced exposure to e-commerce, cloud computing, and semiconductor industries. The weighted average P/E ratio is reasonable given the growth prospects of the included companies.

\section{Diversification and Allocation}
While the portfolio is concentrated, it offers diversification across industries and geographies, with exposure to the U.S., China, and Europe.

\section{Risk Management}
Risks are mitigated through diversification across different sectors and markets. Close monitoring of regulatory developments and market conditions is essential.

\section{Expected Performance}
Based on the intrinsic valuations, the portfolio is expected to outperform the market over the long term, driven by the strong fundamentals of the selected companies.

% Conclusion
\chapter{Conclusion}
This portfolio leverages the strengths of Amazon, Alibaba, and ASML, companies with robust business models and significant growth potential. The long-term value investing approach aims to capitalize on market undervaluations, providing substantial returns over time.

% References
\chapter{References}
\bibliographystyle{plain}
\bibliography{references}

% Appendices
\appendix
\chapter{Appendix}
\section{Detailed Financial Statements}
Include detailed financial statements and projections used in the analyses.

\section{DCF Model Calculations}
Provide the complete DCF model calculations for each company.

\end{document}
