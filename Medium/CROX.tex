\documentclass[12pt]{article}

% Packages
\usepackage{geometry}        % Adjust margins
\usepackage{graphicx}        % Include graphics
\usepackage{booktabs}        % Professional-quality tables
\usepackage{hyperref}        % Hyperlinks
\usepackage{amsmath}         % Math typesetting
\usepackage{float}           % Improved interface for floating objects

% Page Setup
\geometry{
    a4paper,
    left=25mm,
    right=25mm,
    top=25mm,
    bottom=25mm,
}

% Document
\begin{document}

% Title
\begin{center}
    {\Large \textbf{Stock Initial Research Template}}\\[1.5em]
    \textbf{Company Name}: \underline{Crocs, Inc.}\\[0.5em]
    \textbf{Ticker Symbol}: \underline{CROX}\\[0.5em]
    \textbf{Date}: \today
\end{center}

\vspace{1em}

% 1. Company Overview
\section*{1. Company Overview}
Crocs, Inc. is an American footwear company renowned for its innovative casual footwear, particularly its distinctive foam clogs. Founded in 2002 and headquartered in Broomfield, Colorado, Crocs designs, manufactures, and markets a variety of footwear products for men, women, and children. The company's products are crafted from Croslite™, a proprietary closed-cell resin that provides exceptional comfort and support. Crocs operates globally, serving customers through wholesale, retail, and e-commerce channels. :contentReference[oaicite:0]{index=0}

% 2. Key Financial Metrics
\section*{2. Key Financial Metrics}
\begin{table}[H]
    \centering
    \begin{tabular}{@{}ll@{}}
        \toprule
        \textbf{Metric}                 & \textbf{Value}                    \\
        \midrule
        Market Capitalization           & \$\underline{6.6 billion}         \\
        Revenue (TTM)                   & \$\underline{4.1 billion}         \\
        Net Income (TTM)                & \$\underline{792 million}         \\
        Earnings Per Share (EPS)        & \$\underline{13.15}               \\
        Price-to-Earnings (P/E) Ratio   & \underline{8.0}                   \\
        Price-to-Book (P/B) Ratio       & \underline{4.5}                   \\
        Return on Equity (ROE)          & \underline{45.5} \%               \\
        Debt-to-Equity Ratio            & \underline{1.22}                  \\
        Current Ratio                   & \underline{1.8}                   \\
        Dividend Yield                  & \underline{N/A} \%                \\
        \bottomrule
    \end{tabular}
    \caption{Key Financial Metrics}
\end{table}

% 3. Investment Thesis Summary
\section*{3. Investment Thesis Summary}
Crocs has demonstrated robust financial performance, with revenues nearing \$4 billion in 2023, reflecting over 11\% growth. The company's strategic collaborations and strong brand presence have contributed to its success. However, the acquisition of HEYDUDE in 2022 has presented challenges, with the brand experiencing a 14.5\% decline in revenue, impacting overall sales growth. :contentReference[oaicite:1]{index=1}

% 4. Potential Risks and Problems
\section*{4. Potential Risks and Problems}
\begin{itemize}
    \item Declining revenues in the HEYDUDE brand, which may affect overall financial performance.
    \item Legal challenges, such as the \$1.9 million lawsuit from the Federal Trade Commission against HEYDUDE for allegedly hiding negative product reviews. :contentReference[oaicite:2]{index=2}
    \item Operational issues, including shipping delays and refund problems associated with HEYDUDE.
    \item Market competition and changing consumer preferences in the casual footwear industry.
\end{itemize}

% 5. Valuation Snapshot
\section*{5. Valuation Snapshot}
Crocs' stock is trading at a P/E ratio of 8.0, which is lower than industry peers such as Deckers Outdoor, indicating a potentially undervalued position. The company's strong financial metrics, including a high ROE and manageable debt levels, suggest a solid financial foundation. :contentReference[oaicite:3]{index=3}

% 6. Potential and Risks 
\section*{6. Potential and Risks}
Thinking about the P/E ratio and how the actual Crocs brand is doing it could maybe be double the price but the HEYDUDE brand which is 20 percent of the business constitutes a big problem now sales have to normalize for it to look better. Another risk is what if the crocs brand loses market share than the stock is at valuation or more.

% 6. Conclusion
\section*{7. Conclusion}
Given Crocs' strong brand presence and financial performance, the stock warrants further detailed analysis. However, potential investors should closely monitor the performance of the HEYDUDE brand and any associated legal or operational challenges.

% Optional: Additional Notes
\section*{Additional Notes}
\begin{itemize}
    \item Crocs has been proactive in addressing challenges with the HEYDUDE brand, including new marketing investments to enhance brand relevance.
    \item The company's global expansion and product diversification strategies continue to support its growth trajectory.
\end{itemize}

\end{document}
