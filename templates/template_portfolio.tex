\documentclass[12pt]{report}

% Packages
\usepackage{geometry}        % Adjust margins
\usepackage{graphicx}        % Include graphics
\usepackage{amsmath}         % Advanced math typesetting
\usepackage{booktabs}        % Professional-quality tables
\usepackage{hyperref}        % Hyperlinks
\usepackage{fancyhdr}        % Custom headers and footers
\usepackage{longtable}       % Tables that span multiple pages
\usepackage{titlesec}        % Customize section titles
\usepackage{tocloft}         % Customize table of contents
\usepackage{chngcntr}        % Change counter for figures and tables

% Page Setup
\geometry{
    a4paper,
    left=25mm,
    right=25mm,
    top=25mm,
    bottom=25mm,
}

% Header and Footer
\pagestyle{fancy}
\fancyhf{}
\lhead{Long-Term Value Investing Portfolio}
\rhead{\thepage}
\renewcommand{\headrulewidth}{0.4pt}

% Section Formatting
\titleformat{\chapter}{\LARGE\bfseries}{\thechapter}{1em}{}
\titleformat{\section}{\large\bfseries}{\thesection}{1em}{}
\titleformat{\subsection}{\normalsize\bfseries}{\thesubsection}{1em}{}

% Table of Contents Formatting
\renewcommand{\cftchapleader}{\cftdotfill{\cftdotsep}}
\setlength{\cftbeforechapskip}{1.0em}

% Reset figure and table counters for each chapter
\counterwithin{figure}{chapter}
\counterwithin{table}{chapter}

% Document
\begin{document}

% Title Page
\begin{titlepage}
    \centering
    \vspace*{2cm}
    {\huge\bfseries Long-Term Value Investing Portfolio Analysis\par}
    \vspace{1.5cm}
    {\Large Your Name\par}
    \vfill
    {\large \today\par}
\end{titlepage}

% Abstract
\begin{abstract}
\noindent
A concise summary of your portfolio analysis, highlighting your investment philosophy, key portfolio components, and expected outcomes.
\end{abstract}

% Table of Contents
\tableofcontents
\newpage

% Introduction
\chapter{Introduction}
\section{Investment Philosophy}
Describe your long-term value investing philosophy and how it shapes your investment decisions.

\section{Portfolio Overview}
Provide an overview of your concentrated portfolio, including your objectives and the rationale behind selecting these 10 stocks.

% Stock Analyses
\chapter{Stock Analyses}

% Custom Command for Stock Sections
\newcommand{\stock}[2]{
    \chapter{#1}
    \section{Company Overview}
    Offer background information on #1, including its history, business model, and revenue streams.

    \section{Industry Analysis}
    Examine the industry landscape for #1, including market trends and the company's competitive position.

    \section{Investment Thesis}
    Clearly articulate why #1 is a strong candidate for your portfolio.

    \section{Financial Analysis}
    \subsection{Historical Financial Performance}
    Analyze #1's past financial statements, focusing on trends in revenue, net income, and cash flow.

    \subsection{Key Financial Ratios}
    Present and interpret key financial ratios for #1, such as P/E, ROE, ROA, and debt-to-equity.

    \section{Intrinsic Valuation}
    \subsection{Discounted Cash Flow (DCF) Analysis}
    Detail your DCF model for #1, including assumptions, calculations, and results.

    \subsection{Comparable Company Analysis}
    Compare #1's valuation metrics with those of its peers.

    \section{Risk Analysis}
    Identify and discuss the primary risks associated with investing in #1.

    \section{Conclusion}
    Summarize your findings for #1 and restate your investment rationale.
}

% Include analyses for each stock
\stock{Stock 1 Name}{Ticker1}
\stock{Stock 2 Name}{Ticker2}
\stock{Stock 3 Name}{Ticker3}
\stock{Stock 4 Name}{Ticker4}
\stock{Stock 5 Name}{Ticker5}
\stock{Stock 6 Name}{Ticker6}
\stock{Stock 7 Name}{Ticker7}
\stock{Stock 8 Name}{Ticker8}
\stock{Stock 9 Name}{Ticker9}
\stock{Stock 10 Name}{Ticker10}

% Portfolio Analysis
\chapter{Portfolio Analysis}
\section{Aggregate Portfolio Metrics}
Discuss overall portfolio metrics, such as weighted average P/E ratio, dividend yield, and expected growth rates.

\section{Diversification and Allocation}
Analyze how your portfolio is diversified across sectors, industries, or geographies.

\section{Risk Management}
Explain your strategies for managing risks at the portfolio level.

\section{Expected Performance}
Project the expected performance of your portfolio based on your analyses.

% Conclusion
\chapter{Conclusion}
Summarize your investment strategy, key insights from your analyses, and the anticipated benefits of your long-term, value-focused approach.

% References
\chapter{References}
\bibliographystyle{plain}
\bibliography{references}

% Appendices
\appendix
\chapter{Appendix}
Include supplementary materials like detailed financial models, additional charts, or raw data that support your analyses.

\end{document}
